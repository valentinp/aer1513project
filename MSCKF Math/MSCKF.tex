\documentclass[10pt,letterpaper,fleqn,oneside]{article}

% Packages
\usepackage{sectsty}
\usepackage{caption}
%\usepackage{pslatex}
\usepackage{times}
\usepackage{amssymb,amsfonts,amsmath,amscd}
\usepackage[pdftex,colorlinks]{hyperref}
\usepackage[pdftex]{graphicx}
\usepackage{fnpos}
\usepackage{subfigure}
\usepackage{verbatim}
\usepackage{fancyhdr}
\usepackage{theorem}
\usepackage{float}

% Some handy commands
\newcommand{\norm}[1]{\left\Vert#1\right\Vert}
\newcommand{\abs}[1]{\left\vert#1\right\vert}
\newcommand{\set}[1]{\left\{#1\right\}}
\newcommand{\Real}{\mathbb R}
\newcommand{\Complex}{\mathbb C}
\newcommand{\eps}{\varepsilon}
\newcommand{\To}{\longrightarrow}
\newcommand{\Ker}{\textup{Ker}}
\newcommand{\Img}{\textup{Img}}
\newcommand{\diag}{\textup{diag}}
\newcommand{\circulant}{\textup{circ}}
\newcommand{\nl}{\\[0.5em]}

% Tim's Math Defs
\newcommand{\bcf}{\;\mbox{\boldmath ${\cal F}$\unboldmath}}
\def\Vec#1{\mathbf{#1}} %
\newcommand{\bbm}{\begin{bmatrix}}
\newcommand{\ebm}{\end{bmatrix}}
\DeclareMathAlphabet{\mbf}{OT1}{ptm}{b}{n}
\newcommand{\mbs}[1]{{\boldsymbol{#1}}}
\def\ep{\epsilon}
\def\la{\lambda}
\def\om{\omega}
\newcommand{\qc}[1]{{#1}^+}
\newcommand{\qo}[1]{{#1}^\oplus}
\newcommand{\qi}[1]{{#1}^{\scalebox{0.5}{$-1$}}}
\def\four#1#2#3#4#5{\left.^{#2}_{#3}#1^{#4}_{#5}\right.} %

\floatstyle{ruled}
\newfloat{algorithm}{htp}{loa}
\floatname{algorithm}{\sf\bfseries Algorithm}

% Some new environments
\newtheorem{preexample}{\sf\bfseries Example}
\newenvironment{example}
    {\begin{preexample}\upshape\hspace{-5.9pt}{\sf\bfseries :}}
    {~\hfill $\blacktriangleleft$ \end{preexample}}

\newtheorem{prethm}{\sf\bfseries Theorem}
\newenvironment{thm}[1]
    {\begin{prethm}\hspace{-5.9pt}{\sf\bfseries #1:}}
    {\end{prethm}}

\newtheorem{prelem}{\sf\bfseries Lemma}
\newenvironment{lem}[1]
    {\begin{prelem}\hspace{-5.9pt}{\sf\bfseries #1:}}
    {\end{prelem}}

\newtheorem{preprop}{\sf\bfseries Proposition}
\newenvironment{prop}[1]
    {\begin{preprop}\hspace{-5.9pt}{\sf\bfseries #1:}}
    {\end{preprop}}

\newenvironment{prf}[1]
    {\indent\textit{Proof#1:}}
    {~\hfill $\blacksquare$\\}

% Fix some error reporting
\vfuzz2pt % Don't report over-full v-boxes if over-edge is small
\hfuzz2pt % Don't report over-full h-boxes if over-edge is small

% Set the page size
\addtolength{\hoffset}{-1.0in} \addtolength{\voffset}{-0.75in}
\setlength{\textwidth}{7in} \setlength{\textheight}{8.25in}
\setlength{\headheight}{0.6in}
\setlength{\headsep}{0.4in}

\setlength{\footskip}{40pt}
\setlength{\fboxsep}{12pt}

% Set the paragraph ski
\setlength{\parskip}{3pt}

% - User Input --------------------------------------------------
%
% Insert the program, document title, author, etc.

\newcommand{\UTIASprogram}{N/A}
\newcommand{\UTIAStitle}{UTIAS PDF\LaTeX \hspace{0.05in} Report Template \\ \vspace{0.1in} \large{with PDF Graphics from IPE}}
\newcommand{\UTIASdocument}{Multi-State Constraint Kalman Filter: The Math}
\newcommand{\UTIASrevision}{Rev: 1.0}
\newcommand{\UTIASauthor}{Lee Clement and Valentin Peretroukhin}
\newcommand{\UTIASreviewer}{}

% ---------------------------------------------------------------

% PDF setup
\hypersetup{%
    pdftitle={\UTIASdocument: \UTIAStitle},
    pdfauthor={\UTIASauthor},
    pdfkeywords={},
    pdfsubject={\UTIASprogram},
    pdfstartview={},
    urlcolor=cyan,
    linkcolor=red,
}%

% Define some hyphenation
\hyphenation{aero-space} \hyphenation{auton-omous}



% Fix the footnotes location
\makeFNbottom \makeFNbelow

% ---------------------------------------------------------------
\begin{document}

% Define the basic page style
\fancypagestyle{plain}{%
    \fancyhf{}%
    \fancyfoot[C]{}%
    \fancyhead[R]{\begin{tabular}[b]{r}\small\sf \UTIASdocument\\
        \small\sf\UTIASrevision\\
        \small\sf\today \end{tabular}}%
    \fancyhead[L]{\includegraphics[height=0.45in]%
        {figs/utias.pdf} \begin{tabular}[b]{l}\sc{University of Toronto} \\ \sc{Institute for Aerospace Studies} \\ \mbox{} \end{tabular}}%
    \renewcommand{\headrulewidth}{0pt}
    \renewcommand{\footrulewidth}{0pt}}

% Set the page style for the document
\pagestyle{fancy}

% Set the section labeling font
\allsectionsfont{\sf\bfseries}

% Set the caption labeling font
\renewcommand{\captionlabelfont}{\sf\bfseries}

\title{ \vspace{2in}{\sf\bfseries \UTIAStitle} \\
        \vspace{0.5in} \large\normalfont
       Somebody and Somebody Else\\
       Institute for Aerospace Studies \\
       University of Toronto\\
       4925 Dufferin Street \\
       Toronto, Ontario \\
       Canada M3H 5T6 \\
       \texttt{<somebody@utoronto.ca>} \\
       \texttt{<somebodyelse@utoronto.ca>}
}

\author{}
\date{}
%{\sf \maketitle}

% ---------------------------------------------------------------
\newpage

% Set the page headers and footers.
\lhead{ \includegraphics[height=0.45in]%
        {figs/utias.pdf} \begin{tabular}[b]{l}\sc{University of Toronto} \\ \sc{Institute for Aerospace Studies} \\ \mbox{} \end{tabular}}
\rhead{ \begin{tabular}[b]{r}\small\sf \UTIASdocument\\
        \small\sf\UTIASrevision\\
        \small\sf\today \end{tabular}}
\chead{}
\lfoot{}
\cfoot{\thepage}
\rfoot{}

\renewcommand{\headrulewidth}{0pt}
\renewcommand{\footrulewidth}{0pt}


% - Tables ------------------------------------------------------

% Generate table of contents
\newpage
\pagenumbering{roman}
\setcounter{page}{1}
%{\sf\tableofcontents}

% Generate a list of figures
%\newpage
%{\sf\listoffigures}

% Generate a list of tables
%\newpage
%{\sf\listoftables}

% Generate a list of algorithms
%\newpage
%{\sf\listof{algorithm}{List of Algorithms}}

% - Start the Document ------------------------------------------

\newpage
\pagenumbering{arabic}
\setcounter{page}{1}

% - User Input Document Text ------------------------------------

\section{State Parametrization}
\paragraph{IMU State at Time k $(16\times1)$}
\begin{align}
\Vec{X}_{I_k} &= \bbm ^{I_k}_G \bar{q}^T \nl
								^G\Vec{p}_k \nl
								^G\Vec{v}_k \nl
								\Vec{b}_{g_k} \nl
								\Vec{b}_{a_k}
						 \ebm 
						 \begin{array}{l}
						 \leftarrow \text{Global to IMU rotation unit quaterion} \nl
						 \leftarrow \text{IMU position in global frame} \nl
						 \leftarrow \text{IMU velocity in global frame} \nl
						 \leftarrow \text{Gyro bias} \nl
						 \leftarrow \text{Accelerometer bias}
						 \end{array}						 
\end{align}

\paragraph{IMU Error State $(15\times1)$}
\begin{align}
\widetilde{\Vec{X}} &= 	\bbm ^G\widetilde{\boldsymbol{\theta}} \nl
										^G\widetilde{\Vec{p}} \nl
										^G\widetilde{\Vec{v}} \nl
										^G\widetilde{\Vec{b}}_g \nl
										^G\widetilde{\Vec{b}}_a
								\ebm
								=
								\bbm
										\boldsymbol{\delta\theta}_I \nl
										^G\Vec{p} - ^G\hat{\Vec{p}} \nl
										^G\Vec{p} - ^G\hat{\Vec{v}} \nl
										\Vec{b}_g - \hat{\Vec{b}}_g \nl
										\Vec{b}_a - \hat{\Vec{b}}_a 
								\ebm
\end{align}
where
\begin{align}
\delta\bar{q} &= ^I_G\hat{\bar{q}}^{-1} \otimes ^I_G\bar{q} \simeq \bbm \frac{1}{2}\boldsymbol{\delta\theta} \nl 1 \ebm
\end{align}

\paragraph{MSCKF State at Time k $((16+7N)\times1)$}
\begin{align}
\hat{\Vec{X}}_k &= \bbm	\hat{\Vec{X}}_{I_k} \nl
										^{C_1}_G\hat{\bar{q}} \nl
										^G\hat{\Vec{p}}_{C_1} \nl
										\vdots \nl
										^{C_N}_G\hat{\bar{q}} \nl
										^G\hat{\Vec{p}}_{C_N} \nl										
								\ebm
								\begin{array}{l}
								\leftarrow \text{IMU State estimate} \nl
								\leftarrow \text{Camera pose 1 orientation estimate} \nl
								\leftarrow \text{Camera pose 1 position estimate} \nl
								\vdots \nl
								\leftarrow \text{Camera pose N orientation estimate} \nl
								\leftarrow \text{Camera pose N position estimate} \nl
								\end{array}		
\end{align}

\paragraph{MSCKF Error State $((15 + 6N) \times 1)$}
\begin{align}
\widetilde{\Vec{X}}_k &= \bbm	\widetilde{\Vec{X}}_{I_k} \nl
											\boldsymbol{\delta\theta}_{C_1} \nl
											^G\widetilde{\Vec{p}}_{C_1} \nl
											\vdots \nl
											\boldsymbol{\delta\theta}_{C_N} \nl
											^G\widetilde{\Vec{p}}_{C_N}
								\ebm
\end{align}

\section{IMU State Estimate Propagation}
Integrate these with RK5 from $t_k$ to $t_k + T$ where $T$ is the sampling period of the IMU:
\begin{align}
^I_G\dot{\hat{\bar{q}}} &= \frac{1}{2}\boldsymbol{\Omega}\left(\boldsymbol{\omega}_m - \hat{\Vec{b}}_a - \Vec{C}\left(^I_G\hat{\bar{q}}\right) \boldsymbol{\omega}_G \right)^I_G\hat{\bar{q}} \nl
\dot{\hat{\Vec{b}}}_g  &= \Vec{0}_{3\times1} \nl
^G\dot{\hat{\Vec{v}}}_I &= \Vec{C}\left(^I_G\hat{\bar{q}}\right)^T\left(\Vec{a}_m - \hat{\Vec{b}}_a\right) - 2\left(\boldsymbol{\omega}_G^\times\right)^G\hat{\Vec{v}}_I - \left(\boldsymbol{\omega}_G^\times\right)^2 \left.^G\hat{\Vec{p}}_I\right. + ^G\Vec{g} \nl
\dot{\hat{\Vec{b}}}_a &= \Vec{0}_{3\times1} \nl
^G\dot{\hat{\Vec{p}}}_I &= ^G\hat{\Vec{v}}_I \nl
\dot{\widetilde{\Vec{X}}}_I &= \Vec{F}\widetilde{\Vec{X}}_I + \Vec{G}\Vec{n}_I
\end{align}
where
\begin{align}
\boldsymbol{\Omega}\left(\boldsymbol{\omega}\right) &= \bbm -\boldsymbol{\omega}^\times & \boldsymbol{\omega} \nl
																										-\boldsymbol{\omega}^T & 0
																								\ebm 
																								\text{ is }4\times4,
\end{align}
$\Vec{C}(\bar{q})$ is the rotation matrix form of the quaternion $\bar{q}$,
\begin{align}
\Vec{F} &= \bbm	-\hat{\boldsymbol{\omega}}^\times & -\Vec{1}_3 & \Vec{0}_{3\times3} & \Vec{0}_{3\times3} & \Vec{0}_{3\times3} \nl
					\Vec{0}_{3\times3} & \Vec{0}_{3\times3} & \Vec{0}_{3\times3} & \Vec{0}_{3\times3} & \Vec{0}_{3\times3} \nl
					-\Vec{C}\left(^I_G\hat{\bar{q}}\right)^T\hat{\Vec{a}}^\times & \Vec{0}_{3\times3} & -2\boldsymbol{\omega}_G^\times & -\Vec{C}\left(^I_G\hat{\bar{q}}\right)^T & -\left(\boldsymbol{\omega}_G^\times\right)^2 \nl
					\Vec{0}_{3\times3} & \Vec{0}_{3\times3} & \Vec{0}_{3\times3} & \Vec{0}_{3\times3} & \Vec{0}_{3\times3} \nl
					\Vec{0}_{3\times3} & \Vec{0}_{3\times3} & \Vec{1}_3 & \Vec{0}_{3\times3} & \Vec{0}_{3\times3} \nl
		\ebm \text{ is } 15\times15, \nl
\Vec{G} &= 	\bbm	-\Vec{1}_3 & \Vec{0}_{3\times3} & \Vec{0}_{3\times3} & \Vec{0}_{3\times3} \nl
								\Vec{0}_{3\times3} & \Vec{1}_3 & \Vec{0}_{3\times3} & \Vec{0}_{3\times3} \nl
								\Vec{0}_{3\times3} & \Vec{0}_{3\times3} & -\Vec{C}\left(^I_G\hat{\bar{q}}\right) & \Vec{0}_{3\times3} \nl
								\Vec{0}_{3\times3} & \Vec{0}_{3\times3} & \Vec{0}_{3\times3} & \Vec{1}_3 \nl
								\Vec{0}_{3\times3} & \Vec{0}_{3\times3} & \Vec{0}_{3\times3} & \Vec{0}_{3\times3}
					\ebm \text{ is } 15\times12,
\end{align}
and
\begin{align}
\Vec{n}_I &= \bbm \Vec{n}_g \nl \Vec{n}_{wg} \nl \Vec{n}_a \nl \Vec{n}_{wa} \ebm
\end{align}
is the system noise whose covariance matrix $\Vec{Q}_I$ is computed offline during calibration.
\newpage\noindent
For the covariance,
\begin{align}
\Vec{P}_{k+1 | k} &= \bbm	\Vec{P}_{II_{k+1|k}} & \boldsymbol{\Phi}\left(t_k + T, t_k\right)\Vec{P}_{IC_{k|k}}\nl
											\Vec{P}_{IC_{k|k}}^T\boldsymbol{\Phi}\left(t_k + T, t_k\right)^T & \Vec{P}_{CC_{k|k}}\nl
								\ebm \text{ is }\left(15 + 6N\right)\times\left(15 + 6N\right)
\end{align}
where:
\begin{itemize}
\item $\Vec{P}_{CC_{k|k}}$ is the $6N\times6N$ covariance matrix of the camera pose estimates (see section on State Augmentation), 
\item $\Vec{P}_{IC_{k|k}}$ is the correlation between the errors in the IMU state and the camera pose estimates,
\item the state transition matrix $\boldsymbol{\Phi}\left(t_k + T, t_k\right)$ is computed by integrating (with RK5)
\begin{align}
\dot{\boldsymbol{\Phi}}\left(t_k + \tau, t_k\right) &= \Vec{F}\boldsymbol{\Phi}\left(t_k + \tau, t_k\right), &\tau\in\left[0,T\right]
\end{align}
with initial condition $\boldsymbol{\Phi}\left(t_k, t_k\right) = \Vec{1}_{15}$, and
\item $\Vec{P}_{II_{k+1|k}}$ is obtained by integrating (with RK5)
\begin{align}
\dot{\Vec{P}}_{II} &= \Vec{F}\Vec{P}_{II} + \Vec{P}_{II}\Vec{F}^T + \Vec{G}\Vec{Q}_I\Vec{G}^T
\end{align}
over the interval $\left(t_k, t_k+T\right)$.
\end{itemize}

\section{State Augmentation}
\subsection{Camera Poses}
For the $\left(N+1\right)^{\text{th}}$ image, the camera pose is estimated as
\begin{align}
^C_G\hat{\bar{q}} &= \left.^C_I\bar{q}\right.\otimes\left.^I_G\hat{\bar{q}}\right. \nl
^G\hat{\Vec{p}}_C &= \left.^G\hat{\Vec{p}}_I\right. + \Vec{C}\left(^I_G\hat{\bar{q}}\right)^T \left.^I\Vec{p}_C\right.
\end{align}
and the EKF covariance matrix is augmented according to
\begin{align}
\Vec{P}_{k|k} &\leftarrow \bbm \Vec{1}_{15+6N}  \nl \Vec{J}\ebm \Vec{P}_{k|k} \bbm \Vec{1}_{15+6N}  \nl \Vec{J}\ebm^T
\end{align}
where the Jacobian
\begin{align}
\Vec{J} &= \bbm \Vec{C}\left(^C_I\hat{\bar{q}}\right) & \Vec{0}_{3\times9} & \Vec{0}_{3\times3} & \Vec{0}_{3\times6N} \nl
							\left(\Vec{C}\left(^I_G\hat{\bar{q}}\right)^T \left.^I\Vec{p}_C\right.\right)^\times & \Vec{0}_{3\times9} & \Vec{I}_3 & \Vec{0}_{3\times6N}
				\ebm \text{ is } 6\times\left(15 + 6N\right)
\end{align}

\newpage
\section{Correction Equations}
\paragraph{Residuals}
\begin{align}
	\Vec{r}_n &= \Vec{T}_H \widetilde{\Vec{X}} + \Vec{n}_n
\end{align}
where $\Vec{T}_H$ (an upper-triangular matrix) is obtained from the QR decomposition of $\Vec{H}_o$ ($\Vec{H}_\Vec{x}$ in the original tech report, but I think that's an error...)
\begin{align}
\Vec{H}_o &= \bbm	\Vec{H}^{(1)}_o \nl \vdots \nl \Vec{H}^{(L)}_o \ebm
 = \bbm \Vec{Q}_1 & \Vec{Q}_2 \ebm \bbm \Vec{T}_H \nl \Vec{0} \ebm.
\end{align}
Each $\Vec{H}^{(j)}_o$ is the projection of $\Vec{H}^{(j)}_\Vec{x}$ onto the left nullspace of $\Vec{H}^{(j)}_f$

$T_H$ and $\Vec{r}_n$ can be computed in $O(r^2d)$ time using Givens rotations without having to form $\Vec{Q}_1$ explicitly. [FIGURE OUT HOW TO DO THIS]

Each $\Vec{H}^{(j)}_\Vec{x}$ is a stack of Jacobians $\Vec{H}^{(j)}_{\Vec{x}_i}$ of the $i^\text{th}$ measurement of feature $j$ with respect to the state (only the entries corresponding to pose $i$ are non-zero):
\begin{align}
\Vec{H}^{(j)}_{\Vec{x}_i} &= \bbm \Vec{0}_{2\times15} & \Vec{0}_{2\times6} & \hdots & \Vec{J}^{\left(j\right)}_i \left(^{C_i}\hat{\Vec{X}}_{f_j}\right)^\times & -\Vec{J}^{\left(j\right)}_i \Vec{C}\left(^{C_i}_G\hat{\bar{q}}\right) & \hdots & \Vec{0}_{2\times6} \ebm \text{ is } 2\times\left(15 + 6N\right)
\end{align}

Each $\Vec{H}^{(j)}_f$ is a stack of Jacobians $\Vec{H}^{(j)}_{f_i}$ of the $i^\text{th}$ measurement of feature $j$ with respect to the feature position:
\begin{align}
\Vec{H}^{(j)}_{f_i} &= \Vec{J}^{\left(j\right)}_i \Vec{C}\left(^{C_i}_G\hat{\bar{q}}\right) \text{ is } 2\times3
\end{align}

In both equations,
\begin{align}
\Vec{J}^{\left(j\right)}_i &= \frac{1}{^{C_i}\hat{Z}_j}
\bbm 1 & 0 & -\frac{^{C_i}\hat{X}_j}{^{C_i}\hat{Z}_j} \nl
		 0 & 1 & -\frac{^{C_i}\hat{Y}_j}{^{C_i}\hat{Z}_j}
\ebm
\end{align}
where 
\begin{align}
\bbm ^{C_i}\hat{X}_j \nl ^{C_i}\hat{Y}_j \nl ^{C_i}\hat{Z}_j \ebm &= \Vec{C}\left(^{C_i}_G\hat{\bar{q}}\right) \left(^G\hat{\Vec{p}}_{f_j} - ^G\hat{\Vec{p}}_{C_i} \right)
\end{align}
are the 3D coordinates of feature $j$ in the frame of image $i$.

$\Vec{n}_n$ is a noise vector with covariance matrix
\begin{align}
\Vec{R}_n &= \sigma^2_{\text{im}} \Vec{1}_r
\end{align}
where $r$ is the number of columns in $\Vec{Q}_1$.
\newpage
\paragraph{Kalman Gain}
\begin{align}
\Vec{K} &= \Vec{P}_{k+1|k}\Vec{T}_H^T\left(\Vec{T}_H\Vec{P}_{k+1|k}\Vec{T}_H^T + \Vec{R}_n \right)^{-1}
\end{align}

\paragraph{State Vector Correction}
\begin{align}
\Delta\Vec{X} &= \Vec{K}\Vec{r}_n
\end{align}

\paragraph{State Covariance Correction}
\begin{align}
\Vec{P}_{k+1|k+1} &= \left(\Vec{1}_{15+6N} - \Vec{K}\Vec{T}_H \right)\Vec{P}_{k+1|k}\left(\Vec{1}_{15+6N} - \Vec{K}\Vec{T}_H \right)^T + \Vec{K}\Vec{R}_n\Vec{K}^T
\end{align}

% ---------------------------------------------------------------

%\newpage
%\bibliographystyle{plain}
%\bibliography{bib/refs}
% ---------------------------------------------------------------

%\newpage
%\appendix
%\section{Appendix Title}
%\subsection{Subsection of Appendix}

\end{document}
